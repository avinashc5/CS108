\documentclass{article}
\usepackage{graphicx}% Required for inserting images
\graphicspath{{./images/}}



\title{Conic Sections}
\author{23b1064}
\date{\today}

\begin{document}

% preamble
\maketitle
% below line auto generates the table of contents
\tableofcontents
\clearpage

% section --1
\section{Introduction}
% Introduction section %
A conic section, conic or a quadratic curve is a curve obtained from a cone’s surface intersecting a plane. The conic sections in the Euclidean plane have various distinguishing properties, many of which can be used as alternative definitions.
% paragraph
\section{Types of Conic Section}
% section --2
This section explains two types of conic sections.
\begin{itemize}
	\item \textbf{Ellipse}\\
	\begin{figure}[h]
		\centering
		\includegraphics[width=0.35\textwidth]{Ellipse.png}
		\label{ellipse}
		\caption[short]{Ellipse}
	\end{figure}\\
	An ellipse is a plane curve surrounding two focal points, such that for all
	points on the curve, the sum of the two distances to the focal points is a
	constant
	\item \textbf{Parabola}\\
	\begin{figure}[h]
		\centering
		\includegraphics*[width=0.3\textwidth]{parabola.png}
		\label{parabola}
		\caption[short]{Parabola}
	\end{figure}\\
	A parabola is a plane curve which is mirror-symmetrical and is approximately U-shaped
\end{itemize}

% Types of conic section with lists and figures
% paragraph
% unordered list containing figures and content

% section --3
\section{Properties}
This section contains the equations for various conic sections and various parameter values.
% Properties section
%paragraph
% Need to use either equation or display math for equations
% subsection --3.1
\subsection{Equations}
\begin{itemize}
	\item \textbf{Ellipse}
	The equation for ellipse in figure 1 is 
	\begin{equation}
		\frac{x^2}{a^2} + \frac{y^2}{b^2} = 1
	\end{equation}
	\item \textbf{Parabola}
	The equation for ellipse in figure 2 is
	\begin{equation}
		y^2 = 4ax
	\end{equation}
	with $a > 0$
\end{itemize}
% unordered list of equations and content
% subsection --3.2
\subsection{Parameters}
\begin{tabular}{|c|c|c|}
	\hline
	Conic section type & Eccentricity & Semilatus rectum \\
	\hline
	Ellipse & $\sqrt{1-\frac{b^2}{a^2}}$ & $\frac{b^2}{a}$ \\
	Parabola & 1 & $2a$ \\
	\hline
\end{tabular}
% tablular code

\end{document}

